\documentclass[12pt]{article}
\usepackage{graphicx}
\usepackage{cite}
\usepackage{url}
\begin{document}
\title{Linux project: Wine}
\author{Tom Geerits}
\date{Schooljaar 2011-2012}
\maketitle
\newpage
\tableofcontents
\newpage
\listoffigures
\listoftables
\newpage
\section{Wat is Wine?}
Wine is een opensourceprogramma om software die voor Windows gemaakt is, ook te gebruiken op andere besturingssystemen. Dit wordt vaak gebruikt voor Mac of Linux. Wine implementeert zowel 16- als 32 bit Windowsprogramma's, maar kan Windows zelf niet uitvoeren.
\subsection{De naam 'Wine'}
De originele betekenis van 'Wine' was de acroniem voor '\textbf{WIN}dows \textbf{E}mulator'. \\
Deze betekenis is later echter veranderd omdat men dit programma wou onderscheiden van andere 'echte' emulators. Nu betekent 'Wine' niets anders dan \textit{Wine Is Not An Emulator}.
Een emulator maakt het mogelijk om een ander of verouderd computersysteem in een eigen omgeving na te bootsen. Denk maar aan VirtualBox of VMWare.

Dit is echter voor discussie vatbaar, omdat een emulatie in ruimere zin 'het simuleren van een omgeving bovenop de huidige omgeving' kan betekenen. In dit geval is Wine dus wel een emulator en was de vorige betekenis van de naam eigenlijk beter gekozen.
\newpage
\section{Ontstaan van Wine}
In 1993 is Wine opgericht door Bob Amstadt en Eric Youngdale, zodat Windows programma's ook konden uitgevoerd worden onder Linux. Het is gebaseerd op 2 programma's van Sun Microsystems: Wabi en Public Windows Initiative\footnote{Interface die gecreëerd is om een neutraal, platform onafhankelijke standaard op te richting die niet bestuurd wordt door Microsoft}. Oorspronkelijk richtte Wine zich alleen op de 16-bit software, maar sinds 2010 wordt er vooral gekeken naar 32- en 64 bit applicaties. Sinds 1994 leidt Alexandre Juilliard het project.
\subsection{Moeilijkheden bij het programmeren}
Doordat de documentatie van de Windows API vaak onvolledig en onjuist was, heeft het voor de programmeurs veel tijd gekost om dit te maken. Microsoft documenteert zijn 32-bit applicaties zeer goed, maar op sommige vlakken, zoals bij extensies en protocollen, hebben ze geen specifieke Microsoft regel. Dit maakt het voor programmeurs lastig omdat Wine ervoor moet zorgen dat dit alles op een ander besturingssysteem wel perfect werkt.
\subsection{Verschillende versies}
De eerste officiële versie, versie 0.9, werd gelanceerd op 25 oktober 2005 en versie 1.0 werd gelanceerd op 17 juni 2008, na 15 jaar programmeerwerk. 
Sinds versie 1.4 gelanceerd is, zijn er heel wat verbeteringen aangebracht, onder andere vele grafische functies, verschillen in audio en ondersteuning voor applicaties zoals Microsoft Office 2010.
\newpage
\section{Functionaliteit}
De programmeurs die instaan voor het Direct3D gedeelte van Wine, hebben nieuwe functies zoals pixel shaders \footnote{Onderzoeken de kleur en attributen van een pixel, waardoor er speciale effecten aan deze pixel gegeven kunnen worden.} ontworpen. Wine kan bijhorende DLL's ook gebruiken waardoor de functionaliteit verhoogt, maar hiervoor is een licentie van Microsoft nodig. Dit is niet nodig als de DLL's ontworpen zijn samen met de applicatie zelf.

Winecfg is een GUI die ontworpen is om Wine op een gemakkelijke manier te configurereren zonder dat men parameters moet veranderen in het register. Moest dit toch nodig zijn, kan men de nodige parameters handmatig veranderen in een bijgeleverde register editor die lijkt op regedit van Windows. Wine heeft ook standaard programma's als Kladblok, Wordpad en Internet Explorer aan boord.

\subsection{Terugkerende compatibiliteit}
Wine heeft een zeer goede terugkerende compatibiliteit. Het kan verschillende Windows programma's nabootsen, teruggaande tot Windows versie 2.0. Sinds Wine v1.3 is er Windows 1.0 en 2.0 wel uit genomen. Je kan nog steeds "Windows 2.0" selecteren, maar Wine zal gewoon niet reageren dan.

De terugkerende compatibiliteit van Wine is beter als die van Windows. Windows verplicht gebruikers er ooit toe om programma updates te doen bij het vernieuwen van het besturingssysteem, en dat doet Wine niet, omdat Wine een ingebouwde 'compatibility mode' heeft. Zo kan Wine bijvoorbeeld 16 bit programma's draaien op een 64 bit besturingssysteem. 64-bit versies van Windows daarentegen kunnen geen 16 bit programma's draaien.

\subsection{Third-party software}
Third-party software zijn programma's die niet gemaakt zijn door Wine en ook niet door Microsoft. Ze hebben een andere manier van benadering nodig. In dat geval moeten bijvoorbeeld sommige Windows DLL's handmatig geconfigureerd worden. Dit zijn dingen die Wine niet echt ondersteunt. Het is voor gebruikers zeer ongemakkelijk om dergelijke programma's aan het werken te krijgen. De bedoeling van dit soort programma's is ook om gebruikt te 
worden buiten Wine en dergelijke programma's om. Toch zijn er al enkele applicaties die compatibel zijn met Wine. Dit zijn enkele voorbeelden (en zijn te vinden op de Wine wiki): 
\begin{itemize}
	\item CrossOver
	\item Bordeaux: Wine GUI configuratie manager dat winelib applicaties kan laten werken, en ondersteunt ook de installatie van sommige third-party software en de installatie van games
	\item Winetricks: script om sommige basis componenten die typisch van Microsoft zijn, te laten werken onder Wine
	\item QT4Wine
	\item Wine-Doors
	\item IEs4Linux
	\item PlayOnLinux
	\item Wineskin
\end{itemize}
\newpage
\section{Installatie van Wine}
Om Wine te installeren in Ubuntu, moeten we in het Software center op zoek gaan naar het Wine programma. Dit doen we door in het zoekveld 'Wine' in te tikken. Als zoekresultaat zullen er meerdere Wine programma's gevonden worden, maar we kiezen 'Wine Windows program loader (wine 1.3)'.
\begin{figure} [!ht]
\begin{center}
	\includegraphics[scale=0.5]{Installeren}
\end{center}
	\caption{zoeken van het Wine programma in het Software center}
\end{figure}

Als we dit aanklikken en we klikken op Install, moet het paswoord van de computer ingegeven worden, en daarna begint Wine automatisch met downloaden.
\newpage
Na het downloaden wordt automatisch een wizard geopend. We vinken aan dat we akkoord gaan met de voorwaarden en drukken op OK. De installatie gaat verder en na enkele ogenblikken is Wine geïnstalleerd.

Als we nu in het zoekveld van de 'Dash home' Wine ingeven, komt het programma daar te staan en kunnen we aan de slag.
\begin{figure} [!ht]
\begin{center}
	\includegraphics[scale=0.5]{Zoeken}
\end{center}
	\caption{Wine is correct ge\"{\i}nstalleerd en klaar voor gebruik}
\end{figure}
\newpage
\section{Vooroordelen over Wine}
\subsection{Wine is traag omdat het een emulator is}
Wine wordt vaak vergeleken met echte emulators als VMWare, VirtualBox,... en daardoor wordt vaak gedacht dat Wine taag is in gebruik. Dit is echter niet waar. Echte emulators bootsen echte besturingssystemen na en gebruiken daardoor CPU, maar dat heeft Wine niet nodig en daardoor is Wine niet traag.
\subsection{Wine kan niet gebruikt worden zonder Windows}
Dit is niet waar. Wine zorgt ervoor dat alle onderdelen van Windows ge\"{\i}mplementeerd worden en Windows dus overbodig wordt in Linux of Mac OS. 

Het is tot nu toe perfect mogelijk om zonder Windows te werken met Wine, maar omdat er nog veel werk is aan Wine zelf, is het voor sommige programma's wel nodig om sommige Windows componenten te gebruiken die Wine zelf nog niet kan bieden. Het aantal programma's dat Windows nog nodig heeft, vermindert met elke nieuwe release...
\subsection{Wine is alleen voor Linux gemaakt}
Dit is volledig incorrect. Wine wordt het meest gebruikt onder Linux, maar wordt ook ondersteund onder MacOS X, FreeBSD en Solaris.

Doordat de meeste ontwikkelaars werken onder Linux, kan het wel gebeuren dat er sneller een nieuwere release is onder Linux dan onder een niet-Linux platform. Dit wordt vaak wel opgelost in de eerstvolgende release voor het juiste programma. 
\subsection{Wine zal nooit volledig klaar zijn}
Grote software projecten zijn nooit volledig af, er komen telkens nieuwe releases uit, en dit om het programma up to date te houden. Wine heeft wel telkens een flinke achterstand doordat bij elke nieuwe Windows release ook nieuwe API oproepen horen, waardoor het telkens opnieuw zoeken is naar de juiste manier om het programma te laten werken.
\section{Ondersteunde programma's}
\subsection{Games die zonder problemen werken}

\begin{center}
\begin{table} [h]
\begin{tabular}{ | l | p{10cm} |}
	\hline
    Programma & Uitleg \\ \hline
   	 World of Warcraft 4.3.x & Online spel waarbij de speler een eigen 	personage cre\"{e}ert en in een virtuele wereld vrijwel vrij kan rondlopen. \\ 			\hline
    The Sims 3 & Videospel waarbij de speler een eigen familie kan cre\"{e}ren. \\ \hline
    Counterstrike & First-person shooter videospel. 2 groepen spelers moeten het tegen mekaar opnemen door gehele tegenpartijen uit te schakelen of 			doelstellingen te behalen. \\
    \hline
\end{tabular}
\caption{Voorbeelden van programma's die zonder problemen werken met Wine}
\end{table}
\end{center}

\subsection{Games en programma's die mits enige configuratie werken}

\begin{center}
\begin{table} [h]
\begin{tabular}{ | l | p{10cm} |}
	\hline
    Programma & Uitleg \\ \hline
   	 EVE Online 7.3x & Online multiplayer spel dat zich afspeelt in een science-fiction wereld. Spelers zijn astronauten die op zoek moeten gaan naar verschillende voorwerpen. \\ 			\hline
    Steam official release &  Programma om games tegen betaling te downloaden en deze vervolgens te kunnen spelen. Men betaalt dus niet langer voor de CD of DVD met daarop de software, maar voor het recht om het spel te spelen.\\ \hline
    Battlefield 2.1x &  Online first person shoorter spel met 64 spelers in twee teams. Het is ook mogelijk om tanks, boten, vliegtuigen en helikopters te gebruiken.\\
    \hline
\end{tabular}
\caption{Voorbeelden van programma's die zonder problemen werken met Wine}
\end{table}
\end{center}

\subsection{Programma's en games waar nog fouten in zitten}
\begin{center}
\begin{table} [h]
\begin{tabular}{ | l | p{10cm} |}
	\hline
    Programma & Uitleg \\ \hline
   	 Photoshop CS3 & Programma om foto's te bewerken. \\ 			\hline
    Call of Duty 4 & Videospel dat zich afspeelt in de Tweede Wereldoorlog waarin de speler meevecht met de geallieerden voor het Amerikaanse leger, het Engelse leger en het Rode leger van de Sovjet-Unie. \\ \hline
    AutoCAD 2008 & Programma dat ontwikkeld is om technische tekeningen te maken. Het biedt ook de mogelijkheid om 3D-modellen te ontwikkelen.\\
    \hline
\end{tabular}
\caption{Voorbeelden van programma's en games die nog niet volledig compatibel zijn met Wine}
\end{table}
\end{center}
\newpage
\newpage
\section{Bijlage 1: installatie van Mutt}
Mutt is een opensource en tekstgebaseerde e-mailclient en is beschikbaar voor Linux en Unix.

Mutt kan niet zelfstandig e-mails versturen, hiervoor moet het steunen op een Mailserver.

Mutt ondersteunt POP3, IMAP en andere mailprotocollen. Mutt is ook erg configureerbaar. Doordat Mutt weinig geheugen vereist en niet-grafisch is, is het programma geschikt voor het gebruik op eenvoudige systemen en via trage verbindingen. Omwille van de krachtige configuratiemogelijkheden is het programma populair in Unixomgevingen.

\subsection{Installatie van Mutt in Ubuntu}
Om Mutt te installeren, geven we in de terminal 'sudo apt-get install mutt' in. De download en installatie start.
\begin{figure} [h]
\begin{center}
	\includegraphics[scale=0.5]{MuttEersteStap}
\end{center}
	\caption{Mutt installeren: eerste stap}
\end{figure}

Als een venster verschijnt voor de eerste configuratie, druk dan op het toetsenbord op de 'pijl naar rechts' toets en vervolgens op enter. Hierna vraagt Mutt wat voor soort configuratie we willen. Zorg dat 'Internet site' geselecteerd is en navigeer dan naar OK. Druk op enter om te bevestigen.

Als er gevraagd wordt om een 'system mail name' in te geven, geef dan het achterste gedeelte van het gebruikte e-mailadres in. In dit voorbeeld zullen we met Gmail werken. We geven dus gmail.com in. Hierna zal de installatie verder gaan.
\begin{figure} [h]
\begin{center}
	\includegraphics[scale=0.5]{MuttServerMailName}
\end{center}
	\caption{Server mail name invoeren voor Mutt}
\end{figure}

De volgende stap is het aanmaken van mappen waar Mutt headers, bodies en certificaten van mails opslaat. Dit gebeurt door volgende commando's als de installatie voltooid is:


mkdir -p ~/.mutt/cache/headers


mkdir ~/.mutt/cache/bodies


touch ~/.mutt/certificates

\begin{figure} [!h]
\begin{center}
	\includegraphics[scale=0.5]{MuttHeadersCertificaten}
\end{center}
	\caption{Locatie van headers, bodies en certificaten aangeven}
\end{figure}

Als dit gebeurd is, sla dan onderstaande configuratie op in je home directory met als naam .muttrc (het . teken niet vergeten!).

\vspace{10mm}

\# A basic .muttrc for use with Gmail

\# Change the following six lines to match your Gmail account details


set imap\_user = "YOUR.EMAIL@gmail.com"
set imap\_pass = "PASSWORD"
set smtp\_url = "smtp://YOUR.EMAIL@smtp.gmail.com:587/"
set smtp\_pass = "PASSWORD"
set from = "YOUR.EMAIL@gmail.com"
set realname = "YOUR NAME"

\# Change the following line to a different editor you prefer.


set editor = "nano"


\# Basic config, you can leave this as is
set folder = "imaps://imap.gmail.com:993"
set spoolfile = "+INBOX"
set imap\_check\_subscribed
set hostname = gmail.com
set mail\_check = 120
set timeout = 300
set imap\_keepalive = 300
set postponed = "+[GMail]/Drafts"
set record = "+[GMail]/Sent Mail"
set header\_cache=\textasciitilde{}/.mutt/cache/headers
set message\_cachedir=\textasciitilde{}/.mutt/cache/bodies
set certificate\_file=\textasciitilde{}/.mutt/certificates
set move = no
set include
set sort = 'threads'
set sort\_aux = 'reverse-last-date-received'
set auto\_tag = yes
ignore "Authentication-Results:"
ignore "DomainKey-Signature:"
ignore "DKIM-Signature:"
hdr\_order Date From To Cc
alternative\_order text/plain text/html *
auto\_view text/html
bind editor  complete-query
bind editor \^{}T complete
bind editor  noop 


\# Gmail-style keyboard shortcuts
macro index,pager y "unset trash\textbackslash{}n " "Gmail archive message"
macro index,pager d "set trash=\textbackslash{}"imaps://imap.googlemail.com/[GMail]/Bin\textbackslash{}"\textbackslash{}n " "Gmail delete message"
macro index,pager gi "=INBOX" "Go to inbox"
macro index,pager ga "=[Gmail]/All Mail" "Go to all mail"
macro index,pager gs "=[Gmail]/Starred" "Go to starred messages"
macro index,pager gd "=[Gmail]/Drafts" "Go to drafts"

\vspace{10mm}

Als dit gebeurd is, verander dan de eerste zes regels zodat de configuratie van je eigen Gmailaccount gebeurd is. Als je het paswoord veld blanco laat, wordt er geen paswoord onthouden en moet dit telkens ingegeven worden bij het opstarten van Mutt. Dit is veel veiliger.

\begin{figure} [h]
\begin{center}
	\includegraphics[scale=0.5]{MuttConfiguratie}
\end{center}
	\caption{Inloggegevens voor Mutt ingeven}
\end{figure}

Mutt is nu volledig ge\"{\i}nstalleerd en is klaar voor gebruik. Om Mutt op te starten, geef 'mutt' in in de terminal. Druk op een toets om verder te gaan als dit gevraagd wordt.

\begin{figure} [!ht]
\begin{center}
	\includegraphics[scale=0.5]{MuttKlaar}
\end{center}
	\caption{Mutt is correct ge\"{\i}nstalleerd en klaar voor gebruik}
\end{figure}
\newpage
\section{Bijlage 2: Opzetten van versiebeheersysteem met Gitt}
\subsection{Installeren van de software}
Om Gitt te kunnen gebruiken, moet het eerst ge\"{\i}nstalleerd worden. Dit doen we door in de terminal het commando "sudo apt-get install git-core git-gui git-doc" in te geven. De terminal zal vragen naar het wachtwoord van de gebruiker. Geef dit in om verder te gaan.
\subsection{Aanmaken SSH sleutel}
Om een veilige verbinding te krijgen tussen GitHub en de computer, mmaakt GitHub gebruik van SSH sleutels. We dienen hiervoor dus een SSH sleutel aan te maken. In de terminal geven we "cd ~/.ssh" in om te kijken of er al een SSH sleutel bestaat op de computer. Als de terminal 'No such file or directory' op het scherm toont, wil dat zeggen dat er nog geen SSH sleutel op de computer staat. In dat geval kunnen we direct een nieuwe sleutel aanmaken. Anders dienen we eerst een back-up van de bestaande sleutel te maken en de sleutel te verwijderen. Dit doen we door de volgende commando's in te geven:


ls


mkdir key\_backup


cp id\_rsa* key\_backup


rm id\_rsa*


Als dit gebeurd is, kunnen we een nieuwe sleutel genereren. We geven "ssh-keygen -t rsa -C "jij@jouwemailadres.com"" in. De computer zal nu een sleutel genereren. Als de sleutel aangemaakt is, zal de terminal vragen om een passphrase in te voeren. Dit is aan te raden, omdat de sleutel dan extra beveiligd is. Als de sleutel gegenereerd is, zou er iets gelijkaardigs aan figuur \ref{ssh} moeten verschijnen op het scherm.

\begin{figure} [!ht]
\begin{center}
	\includegraphics[scale=0.5]{SSHKey}
\end{center}
	\caption{SSH sleutel is aangemaakt}
	\label{ssh}
\end{figure}

Als de sleutel aangemaakt is, moet deze toegevoegd worden aan de GitHub website om een online back-up te maken van de gewijzigde bestanden. Hiervoor dient eerst een account aangemaakt te worden op de website van GitHub (www.github.com). 

Eenmaal aangemeld, klik dan rechtsboen op teken langs de knop om uit te loggen. Vervolgens kiezen we links in het menu 'SSH Keys' en daarna rechtsboven 'Add SSH Key'.

We gaan terug naar de Linux computer en openen de Home folder van de computer en daarna de map .ssh. Hier zou het bestand 'id-rsa.pub' moeten staan. Open dit met een tekst editor om de inhoud kopi\"{e}ren. Als de map of het bestand er niet tussen staat, druk dan CTRL + H in op het toetsenbord om verborgen bestanden weer te geven.

\begin{figure} [!ht]
\begin{center}
	\includegraphics[scale=0.5]{SSHOpenen}
\end{center}
	\caption{SSH sleutel openen met een tekst editor}
\end{figure}

Kopieer de inhoud van dit document volledig en plak het dan op de GitHub website in het 'Key' veld. Klik op 'Add key' om te bevestigen.

Om alles even uit te testen, vullen we in de Terminal het volgende commando in: "ssh -T git@github.com". Het is belangrijk dat git@github.com NIET veranderd wordt!

Als alles correct verlopen is, zal Git je verwelkomen.

\subsection{Persoonlijke informatie aanpassen}
Om alles overzichtelijk te houden, zeker wanneer er met meerdere personen aan 1 project gewerkt wordt, is het belangrijk dat elke gebruiker zijn persoonlijke informatie zo correct mogelijk invult. Dit is mogelijk via de terminal via de volgende commando's:

git config --global user.name "Voornaam Achternaam"

git config --global user.email "jij@jouwemailadres.com"

Als dit gebeurd is, is Git correct geconfigureerd en bijna klaar voor gebruik.

\subsection{Repository aanmaken voor online back-up}
Telkens als er gecommit wordt, slaat Git dit lokaal op. Om een online back-up te hebben, moet er ook online een opslagplaats voorzien worden. Dit doen we door naar de GitHub website te surfen, en rechts op 'New repository' te klikken.

\begin{figure} [!ht]
\begin{center}
	\includegraphics[scale=0.5]{GitRepository}
\end{center}
	\caption{Online repository aanmaken voor Git}
\end{figure}

Vul alle informatie in en klik op 'Create repository'.

Om er in grotere projecten nog aan uit te kunnen hoe alles precies werkt, worden vaak README's gemaakt. Om een readme aan te maken voor je project, typ dan de volgende commando's in de terminal in:

mkdir ~/Projectnaam

cd ~/Projectnaam

git init

touch README

Nu is er een leeg readme bestand aangemaakt op de computer. Dit bestand dient echter nog klaargezet te worden om online opgeslagen te worden. Dit doen we door lokaal te committen. Gebruik de volgende commando's in de terminal:

git add README

git commit -m 'dit is mijn eerste commit'

Om nu dit bestand online op te slaan voeren we onderstaande commando's in. Dit zijn tevens de commando's die gebruikt worden om je project lokaal en online te committen.

git remote add origin git@github.com:gebruikersnaam/Projectnaam.git

git push -u origin master

Let op: het eerste commando dient slechts eenmaal ingevoerd te worden per keer dat de computer connectie heeft met GitHub.

Git is nu geconfigureerd en klaar voor gebruik.
\newpage
\section{Bash script aanmaken}
Het onderstaande bash script zorgt ervoor dat er een email wordt verstuurd met een persoonlijke aanspreking, en deze PDF als bijlage. De naam, voornaam en aanspreektitel worden in het onderwerp mee verstuurd.

\#!/bin/bash

echo -e "Geef de aanspreektitel:"

read Aanspreektitel

echo -e "Wat is de voornaam?"

read Voornaam

echo -e "Wat is de achternaam?"

read Achternaam

echo -m "Wat is het emailadres?"

read Emailadres

echo | mutt -s "\$Aanspreektitel \$Achternaam \$Voornaam" \$Emailadres -a ~/Project.pdf


Eerst worden alle nodige variabelen ingevoerd en vervolgens wordt dit via mail verzonden naar de begunstigde.
\newpage
\nocite{*}
\bibliographystyle{plain}
\bibliography{ref}
\end{document}
